\section{Functional Interface}
Is een interface met een enkele abstracte methode en is bedoeld om op een intuïtieve manier fucties door te geven.

\subsection{Aanroepen}
Het aanroepen van een functionele interface kan op vier verschillende manieren, waar in dezelfde code uitvoeren wordt.
Door een object instantie mee te geven die de interface implementeerd, traditionele methode.
Door een referentie naar een statische methode of een instantie methode.
Of door middel van een lambda expressie.

\subsubsection{Traditioneel}
Sort methode krijgt een geinitiaiseerde klasse van Comparator meegegeven: \texttt{.sort(new Author.CompareByInitials());}.
De Comparator is gedefineerd in de Author klasse: \texttt{public static class CompareByInitials implements Comparator<Author> \{ ... \}}.

\subsubsection{Statische methode}
Vergelijk twee objecten.
Een referentie naar een statische methode mee geven: \texttt{.sort(Author::compareByInitials)}.
\texttt{::} betekend een referentie naar een methode.
De statische methode is gedefineerd in de Author klasse: \texttt{public static int compareByInitials2(Author a1, Author a2) \{ ... \}}.

\subsubsection{Instantie methode}
Vergelijk object met huidige instantie.
Geef een referentie naar een methode van de instantie: \texttt{.sort(Author::compareByInitials)}.
De methode geimplementeerd in de Author klasse: \texttt{public int compareByInitials(Author o) \{ ... \}}

\subsubsection{Lambda expressie}
Vergelijk twee objecten met een lambda expressie: \texttt{.sort\{(a1, a2) -> \{ ... \}\}}.

\subsection{Lambda}
Origneel een wiskundig formalisme in de theoretische informatica.
Het 'bewijst' de uitkomst van berekeningen door manipulatie van functionele uitdrukkingen.

\subsubsection{Java lambda}
In Java wordt lambda gebruikt als compacte notatie om een instantie van een functionele interface te maken.
Ook genoemd als arrow notation of anonymous function.
Het wordt geformateerd als \texttt{(parameter) -> expressie afhankelijk van parameters}.

\subsection{Predicate}
Functionele interface met parameter dat een boolean terug geeft, handig voor onder andere filter criteria.
Methode die een predicate implementeerd: \texttt{public Set <Book> findBooks(Predicate<Book> filter) \{ ... \}}.
Te gebruiken met: \texttt{.findBooks(book -> book.yearOfIssue < 2000)}.

\subsection{Overzicht}
Overzicht van functionele interfaces met omschrijfing en bijpassende methode.
\begin{center}
    \begin{longtable}{lp{5cm}l}
        Interface   & Omschrijving & Methode \\
        \midrule
        Consumer<T>         & \'e\'en argument van type T dat niets terug geeft.        & .accept(t) \\
        Predicate<T>        & \'e\'en argument van type T dat boolean terug geeft.      & .test(t) \\
        BiConsumer<T, U>    & \'e\'en argument van type T en U dat niets terug geeft.   & .test(t, u) \\
        UnaryOperator<T>    & \'e\'en argument van type T dat type T terug geeft.       & .apply(t) \\
        Function<T, R>      & \'e\'en argument van type T en type R terug geeft.        & .apply(t) \\
        BiFUnction<T, U, R> & twee argumenten van type T en U dat type R terug geeft.   & .apply(t, u) \\
        BinaryOperator<T>   & twee argumenten van type T en dat type T terug geeft.     & .apply(t1, t2) \\
        Comparator<T>       & twee argumenten van type T dat integer terug geeft.       & .compare(t1, t2) \\
        Supplier<R>         & Methode zonder argument dat type R terug geeft.           & .get() \\
        \bottomrule
    \end{longtable}
\end{center}

\subsection{Advanced map operators}
\texttt{compute(key, remappingFunction)}: Maakt gebruikt van de geveven transformatie functie en returned de waarde.
\texttt{merge(key, value, operatorFunction)}: Combineerd de bestaande mapped waarde met de gegeven waarde door gebruik te maken meegegeven operator functie, slaat de waarde op en geeft je het resultaat.
\texttt{forEach(action)}: Voert de actie functie uit voor ieder sleutel-waarde paar in de map.
\texttt{replaceAll(remappingFunction)}: vervang alle gemapte velden met het resultaat van het toepassen van de remapping methode on elke sleutel-waarde in de map.

\subsection{Compositie}
Functies combineren.
\textbf{and()}: Combineer twee functies.
\textbf{andThen}: Voer de tweede functie uit op basis van de uitkomst van de eerst functie.

\subsection{Default methods}
Interfaces kunnen default methodes implementen met \texttt{default} keyword, een klasse die de interface implementeren kunnen die methode gebruiken.
Zorgt er voor dat je abstract kan zijn en een default implementatie kan hebben.
Bijvoorbeeld: \texttt{default void print() \{ ... \}}
