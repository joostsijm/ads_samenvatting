\section{datatypes}
Datatypes zoals ArrayList, LinkedList, Stack, Queue, Priority Queue, Heap, Sets, en Maps worden in dit hoofdstuk besproken.
% List, ArrayList, LinkedList, Stack, Queue, Priority Queue, Heap, Sets, Maps

\subsection{List}
Een list is een abstract data type, omdat het op meerdere manier geimplementeerd kan worden
In Java is hier voor een ArrayList en een Linkedlist.

\subsubsection{ArrayList}
Dit is een list gebaseerd op een array, de array houd de informatie vast.
Door middel van een variable \textit{Lenghte} weet je waar het laatste item van je collectie is.
In de construtor van de klasse initialiseer je de array met een zelf bepaalde lengte.
\textbf{Toevoegen} gebeurt door eerst \textit{length} te vergelijken met de lengte van de array.
Wanneer \textit{length} gelijk is maak je een nieuwe array aan met de dubbele lengte van de huidige array.
Daarna kopier je de waardes van de oude array, en vervang je hem.
Vervolgens kan je op de index van \textit{length} de nieuwe waarde toevoegen.
\textbf{Verwijderen}, check eerst of de gegeven index lager is dan \textit{length}.
Verplaats alle elementen die na de gegeven index komen een terug, en verlaag \textit{length} met een.

\subsubsection{LinkedList}
Gebruikt geen array, maar nodes die onderling aan elkaar gelinked zijn.
Deze node is een prive klasse die de waarde vasthoud en een \textit{next} property heeft naar de volgende node.
De list zelf houd een referentie naar de eerste, \textit{head}, en laatste, \textit{tail}, node bij.
\textbf{Add} maak een node aan met de opgegeven waarde.
Controller of \textit{head} bestaat, zo niet verwijst die naar deze node.
Anders wordt de node toevegoed aan de \textit{next} van van \textit{tail}
Vervolgens wordt \textit{tail} vervangen met de nieuwe node.
\textbf{Get} voor het terughalen van een index begin je bij \textit{head} en tel je \textit{next} tot je bij de gewenste index bent.
\textbf{Exists}, loop over de nodes van af \textit{head}, wanneer er een match is return \textit{true}, als er geen match is return \textit{false}.
\textbf{Delete} van af head lopen naar de gewenste node, onthoud de laatst bezichte node. Bij de gewenste index laat de \textit{next} van laatst bezochte node verwijzen naar de \textit{next} van de matchende node.

Er bestaan verschillende types.
\textbf{Singular}: iedere node refereerd naar de volgende node in de list.
\textbf{Double}: iedere node refereerd ook naar de vorige node in de list.
\textbf{Circular}: de laastste node refereerd naar de eerste node in de list.

\subsubsection{Vergelijking}
\textbf{ArrayList} heeft fixed lengte, snelle directe toegang, moeilijker toevoegen en verwijderen, altijd te veel of te weinig geheugen.
\textbf{LinkedList} heeft een variable lengte, makkelijk groeien en verkleinen, geen directe toegang, makkelijk toevoegen / verwijderen, objecten niet uitgelijnd in geheugen.
