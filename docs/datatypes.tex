\section{Datatypes}
Datatypes zoals ArrayList, LinkedList, Stack, Queue, Priority Queue, Heap, Sets, en Maps worden in dit hoofdstuk besproken.
% List, ArrayList, LinkedList, Stack, Queue, Priority Queue, Heap, Sets, Maps

\subsection{List}
Een list is een abstract data type, omdat het op meerdere manier geïmplementeerd kan worden
In Java is hier voor een ArrayList en een Linkedlist.

\subsubsection{ArrayList}
Dit is een list gebaseerd op een array, de array houd de informatie vast.
Door middel van een variable \texttt{Length} weet je waar het laatste item van je collectie is.
In de construtor van de klasse initialiseer je de array met een zelf bepaalde lengte.
\textbf{Toevoegen} gebeurt door eerst \texttt{length} te vergelijken met de lengte van de array.
Wanneer \texttt{length} gelijk is maak je een nieuwe array aan met de dubbele lengte van de huidige array.
Daarna kopier je de waardes van de oude array, en vervang je hem.
Vervolgens kan je op de index van \texttt{length} de nieuwe waarde toevoegen.
\textbf{Verwijderen}, check eerst of de gegeven index lager is dan \texttt{length}.
Verplaats alle elementen die na de gegeven index komen een terug, en verlaag \texttt{length} met een.

\subsubsection{LinkedList}
Gebruikt geen array, maar nodes die onderling aan elkaar gelinked zijn.
Deze node is een prive klasse die de waarde vasthoud en een \texttt{next} property heeft naar de volgende node.
De list zelf houd een referentie naar de eerste, \texttt{head}, en laatste, \texttt{tail}, node bij.
\textbf{Add} maak een node aan met de opgegeven waarde.
Controller of \texttt{head} bestaat, zo niet verwijst die naar deze node.
Anders wordt de node toevegoed aan de \texttt{next} van van \texttt{tail}.
Vervolgens wordt \texttt{tail} vervangen met de nieuwe node.
\textbf{Get} voor het terughalen van een index begin je bij \texttt{head} en tel je \texttt{next} tot je bij de gewenste index bent.
\textbf{Exists}, loop over de nodes van af \texttt{head}, wanneer er een match is return \texttt{true}, als er geen match is return \texttt{false}.
\textbf{Delete} van af head lopen naar de gewenste node, onthoud de laatst bezichte node. Bij de gewenste index laat de \texttt{next} van laatst bezochte node verwijzen naar de \texttt{next} van de matchende node.

Er bestaan verschillende types.
\textbf{Singular}: iedere node refereerd naar de volgende node in de list.
\textbf{Double}: iedere node refereerd ook naar de vorige node in de list.
\textbf{Circular}: de laastste node refereerd naar de eerste node in de list.

\subsubsection{Vergelijking}
\textbf{ArrayList} heeft fixed lengte, snelle directe toegang, moeilijker toevoegen en verwijderen, altijd te veel of te weinig geheugen.
\textbf{LinkedList} heeft een variable lengte, makkelijk groeien en verkleinen, geen directe toegang, makkelijk toevoegen / verwijderen, objecten niet uitgelijnd in geheugen.

\subsection{Stack}
Wat er als eerste in gaat komt er als laatste pas uit, en andersom (FILO, LIFO).
Bijvoorbeeld gebruik: undo en redo, terugkeren laatste kruispunt, recursie bijhouden van frames.
Implementatie kan onder andere met array, arrayList / linkedList.
Operaties:
\textbf{push}: Toevoegen aan het eind.
\textbf{pop}: Afhalen van het eind.
\textbf{peek}: Volgende waarde bekijken.

\subsection{Queue}
Wat er als eerste in gaat komt er als eerste uit (FIFO).
Voorbeelden: print queue, aanvraag verwerken op server.
Met een implementatie met een array heb je twee int variable voor begin en eind van de queue.
Bij enqueue verhoogt de eind variable, bij dequeue verhoogt start variable.
Als eind variable is bereikt zet begin variable op 0 als daar plek is, anders verdubbel je de array in lengte.
\textbf{Deque} is een double ended queue, begin is zowel start als einde.
Heeft daar extra operations voor.
Operations:
\textbf{enqueue}: Toevoegen aan begin.
\textbf{dequeue}: Verwijderen aan het eind.
\textbf{peek}: Volgende waarde bekijken.


\subsubsection{Heap}
Ook wel het zelfde als een priority queue.
Heeft als structuur een binary tree dat is geïmplementeerd met een array.
Eerste item begint op 1, van linker child $index \times 2$, voor rechter child $index \times 2 + 1$, voor parent $\frac{index}{2}$.
Elk pad van af root geeft een ordered list, maar er is geen traveral mogelijk want het is weakly ordered.
Efficientie: toevoegen / verwijderen: $O(\log N)$, zoeken $O(N)$.
\textbf{Verwijderen}: return gewenste index, zet laatste node op deze plek en voer sink op deze index uit.
\textbf{Sink}: vergelijk hoogste child, controlleer of hoogste child hoger is dan de parent, anders ben je klaar.
Verplaats huidige positie met hoogste child en herhaal tot er geen childs meer zijn.
\textbf{Toevoegen}: Voeg element toe aan het einde van array, gebruikt swim methode om goede positie te vinden.
\textbf{Swim}: Als parent hoger is verwissel positie, herhaal tot parent lager is.

\subsection{Set}
Een ongeorderde collectie waar ieder element uniek is.
Bijvoorbeeld studenten in een klas.
Operations:
\textbf{addAll}: union,
\textbf{containsAll}: subset,
\textbf{retainAll}: intersection,
\textbf{removeAll}: minus,

\subsubsection{TreeSet}
Gebruikt \index{boomstructuur}, zoals bij een heap, om elementen op te slaan.
Vergelijkt met bestaande element om te kijken of ze al zijn toegevoegd.
Elementen hebben \index{Comparable} geïmplementeerd, of er moet een \index{Comparator} meegegeven worden.

\subsection{HashSet}
Set op basis van hash van de elementen. 

\subsection{Map}
Is een collectie voor sleutel-waarde abstractie.
Waardes worden opgeslagen met een sleutel en opgevraagd met een sleutel.
Voorbeeld: DNS, geeft ip-adres op basis van een domeinnaam.

\subsubsection{Map interface}
In Java is Map een interface me twee generics, K voor key en V voor value.
Toegang methodes:
\textbf{get}: Geeft waarde op basis van key terug.
\textbf{getOrDefault}: Geeft waarde op basis van key of default waarde als key niet bestaad.
\textbf{put}: Voeg een waarde toe in de map op basis van key.
\textbf{putIfAbsent}: Voeg een waarde toe als die nog nog niet bestaat.
\textbf{remove}: Verwijder waarde op basis van een key.
\textbf{containsKey}: Controlleer of key bestaad.
\textbf{containsValue}: Controlleer of waarde bestaad.
\textbf{keySet}: geeft een set van sleutels.
\textbf{value}: geeft een collectie van waardes.

\subsubsection{TreeMap}
Map geimplementeerd als boom structuur op basis van Comparable of Comparator. 

\subsection{HashMap}
Map op basis van hash van de elementen. 
