\section*{Inleiding}
Deze samenvatting is bedoeld voor het vak Algorithms and Data Structures van de Hogeschool van Amsterdam.
De inhoud is gebaseerd op de vak dat in het leerjaar 2020-2021 is gegeven.
Hier in probeer ik op mijn eigen manier de stof van het vak uit te leggen.
Mijn motivatie voor het schrijven van deze samenvatting is om mijn eigen kennis van het vak te testen.

Het is mijn streven om voor de besproken algorithmes code voorbeelden te geven, zowel als in pseudocode en eventueel echte programmeertalen.
Het formaat van deze PDF samenvatting is A5, zo dat het gemakkelijk te lezen is op bijvoorbeeld een E-Reader.

% OOP principes toepassen.
% Het kiezen van een correct data structuur afhankelijk van de vereisten en correct toepassen.
% Recursieve methodes kunnen schrijven.
% Vertrouwd zijn met de classes en interfaces van de java CollectionFramework.
% Het bepalen van de efficiency van algoritmes.
% Uitleggen hoe enkele sorteer algorithmes werken.
% Gebruik maken van functionele interfaces en lambda expressies.
