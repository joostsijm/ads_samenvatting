\section{OOP}
% OOP (references, wrappers, generics, interfaces), abstract datatypes, interator, Recursion, Comparator, Big-O efficiency
Dit hoofdstuk beschrijft de onderwerpen die onder \textit{object oriented programming} vallen, zoals: references, wrappers, generics, interfaces.
Maar ook enkele geleateerde onderwerpen waar voor geen andere toepasselijk hoefdstuk is. Zoals: abstract datatypes, interator, Recursion, Comparator, Big-O efficiency

\subsection{Objectgeoriënteerd programmeren}
\index{OOP} wordt gebruikt om de echte wereld te modeleren.
Voorbeelden daar van zijn \textbf{luchtverkeersleiding}: vliegtuigen, vliegvelden, vluchten; \textbf{account systemen}: klant, account, transactie; \textbf{tekst verwerker}: letter, woord, lettertype, formaat.
Wordt toegepast voor voor medium tot grote applicatie, goed schaalbaar.

\subsubsection{Systemen}
Objectgeoriënteerde systemen zijn samengesteld, modulair en georganiseerd rond data.
Het is de beste strategie voor grote, evoluerende softwaresystemen.
Een klas representeerd een stuk data.
Het forceert, in de meeste gevallen, ontwerpen voor programmeren.

\subsubsection{Fundamentele eigenschappen}
\textbf{Erfenis} (inheritance): mogelijkheid om af te stammen van andere objecten.
Methodes uit de ouder kunnen overschreven worden, methode behoud de naam, dynamische binding laat polymorphisme toe.
\textbf{Inkapseling} (encapsulation): beperken van controlle over de manier een object gemanipuleerd kan worden.
Private properties, public get, set, en andere methodes.
\textbf{polymorfisme} (polymorphism): vermogen van een object om verschillende vormen aan te nemen.
Zorgt voor minder korte koppeling door erfenis.

\subsubsection{Klassen}
Werken als een sjabloon voor een object.
Specificeert de eigenschappen van een groep vergelijkbare objecten.

\subsubsection{Objecten}
Dit is de fundamentele abstractie waar op we onze systemen bouwen.
Een object is een instantie van een klas in het geheugen.
Het heeft drie karakter eigenschappen: \textbf{Staat}: velden, \textbf{gedrag}: methodes, \textbf{identiteit}: hoe we verwijzen naar object.

\subsubsection{Relaties}
Objecten werken samen en zijn verbonden met andere objecten.
Ideaal minimalizeer je korte koppeling, hoge afhankelijkheid, tussen objecten.
Zonder interactie tussen objecten is er geen werkend systeem.
Mogelijkheden om objecten te verbinden:
\textbf{Associate}: connectie met ander object, eventueel een kardinaliteit, kunnen los van elkaar bestaan.
\textbf{Aggregratie}: een groepering van objecten, object kan tijdelijk deel uitmaken en los bestaan.
\textbf{Compositie}: een samengestelde groep van objecten, kunnen niet los bestaan, hebben een korte koppeling.
\textbf{Erfenis}: object stamt af van een ander object (ouder kind relatie), is een korte koppeling.
Door middel van sub-classing kan er extra functionaliteit worden toegevoegd.

\subsection{Modularisatie}
Hierbij breken we iets, een applicatie, op in kleine beheersbare stukken.
Kleinere onderdelen zijn makkelijker om mee te werken, in een team, en te testen.

\subsection{Abstractie}
We denken hier over iets als een idee, we negeren de irrelevante details en leggen de nadruk op de essentiele details.
Zonder ons zorgen te maken over de details verwijderen we de essentiele dingen die alle soorten gemeen hebben.
Bijvoorbeeld: een abstracte auto bestaat uit een abstracte motor en abstacte wielen, maar bijvoorbeeld kleur onbetekenend voor het idee van een auto.
Abstractie zorgt voor een lage koppeling tussen objecten, veel ontwerp patronen zijn hier op gebaseerd

\subsubsection{Abstract Klas}
Kan niet geïnstantieerd worden, vaak is dit in het ontwerp niet wenselijk of logisch.
Een voorbeeld kan zijn: een vorm, je zou het niet kunnen tekennen zonder het figuur te weten.
Definieert het type en heeft een constructor, een ouder en kan ouder zijn van een (abstracte) klas.
Kan (abstracte) methodes en instance fields hebben.

\subsubsection{Interface}
Functioneert als een belofte van gedrag (methodes), van de klas die het implementeerd.
Is een puur abstracte klas om gedrag te defineren met minimale functionaliteit en heeft geen constructor, alleen abstracte methodes.

\subsubsection{Abstracte data types}
Zijn datatypes war van de implementatie van de methodes verborgen is.
Bevat data specefie voro een instante van de abstracte data type met methodes om deze data te bewerken.
De data is ingekapseld, het toont slechts enkele van de subprogramma's.
Bijvoorbeeld een list pebruikt een array en laat de gebruiker door middel van een interface bewerken.
Voor komt direct bewerken van de indices in de array.

\subsection{Generics}
Java is een statisch getypeerde programmeertaal, types moeten van te voren gedefineerd zijn.
Een generics zorgt er voor dat een abstracte data type geparametriseerd kan worden.
Het zorgt er voor dat het type pas later gespecificeerd kan worden, de generic functioneert als een parameter in de klas declaration.
Zonder zou ieder object in ongecontrolleerd in een collectie toegevoegd kunnen worden.
Bij het ophalen van het object moet het eerst gedowncast worden om te gebruiken.

\subsection{Java collections API}
Java implementatie van abstracte data types, \textit{java.util.Collection}.
Zijn gebaseerd op \textbf{AbstractCollection} en implementeren de \ interface \textbf{Collection}.
Bevat bijna ieder type van collectie.

\begin{center}
    \begin{tabular}{ll}
        Naam        & Omschrijving \\
        \midrule
        ArrayList   & Resizable  array - groeit incrementeel \\
        Hashmap     & Een collectie van [key, data] paren - ongeordend \\
        HashSet     & Set door middel van een hash table - ongeordend \\
        Hashtable   & Een collectie van [key, data] paren - ongeordend \\
        LinkedList  & Een gekoppelde lijst \\
        Stack Last  & Laatste er in, eerste er uit (LIFO) data structuur \\
        TreeMap     & [key,data] paren, gehouden in aflopende volgorde \\
        TreeSet     & Sit geimplementeerd als een tree \\
        \bottomrule
    \end{tabular}
\end{center}

