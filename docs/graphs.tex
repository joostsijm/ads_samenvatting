\section{Graphs}
Dit hoofdstuk beschrijft graven, met onder andere de thema's Minimal Spanning Tree, Path Searching en Graph Search Heuristics.
Hier worden de volgende zoek algorithmes beschreven: Depth First Search, Breath First Search, Dijkstra Shortest Path, A*-search.
% Undirected Graph, Minimal Spanning Tree, Path Searching, Dijkstra Shortest Path, Graph Search Heuristics, A*-search

\subsection{Graaf}
Is een datastructuur met nodes en edges, nodes worden verbonden met edges.
Wordt gebruikt in bijvoorbeel: route kaart, sociale connecties, filesystem, kracht balans, logistiek, beslissingsondersteuning.
Termen:
\textbf{Connect}: Er is een pad mogelijk van elke node naar elke andere node.
\textbf{Disconnected}: Niet iedere node is met een pad te verbinden.
\textbf{Directed}: Edges hebben een richting.
\textbf{Undirected}: Edges kunnen beide kanten.
\textbf{Weighted}: Edges hebben een waarde (afstand, kost).
\textbf{Dense}: Graaf met veel edges tussen de nodes.

\subsection{Path}
\textbf{simple path}: Pad van edges zonder herhalende nodes.
\textbf{Cycle}: Pad waar de start en eind node hetzelfde zijn.

\subsubsection{Euler}
\textbf{Euler path}: Pad waar alle edges een keer gebruikt worden.
Als er exact twee nodes zijn met een oneven degree.
\textbf{Euler cycle}: Pad waar alle edges een keer gebruikt worden met dezelfde start en eind node.
Als alle nodes een even degree hebben.
\textbf{Euler Graph}: Graph met een Euler cycle.
\textbf{Semi-Euler Graph}: Graph met een Euler path maar geen cycle.

\subsubsection{Hamilton}
\textbf{Hamilton path}: Pad waar alle nodes een keer gebruikt worden.
\textbf{Hamilton cycle}: Pad waar alle nodes een keer gebruikt worden met dezelfde start en eind node.
\textbf{Hamilton Graph}: Graph met een Hamilton cycle.
\textbf{Semi-Hamilton Graph}: Graph met een hamilton path maar geen cycle.

\subsection{Traveling salesman problem}
Het is lastig om het kortste pad te vinden tussen alle nodes in een graaf.
Dis is een np-complete probleem, een moeilijk probleem zonder efficiënte oplossing.
Complexiteit bij bestaande oplossing $O(2^N)$, hier bij worden alle mogelijke paden uitgerekend.

\subsection{Spanning tree}
Een spanning tree van een connected graaf is een tree waar in alle nodes zijn gebruikt en alle edges van de tree ook in de graaf voorkomen.

\subsection{Implementatie}
\textbf{Adjecency List}: Een set met sub-lists waar in edge klasses worden bijgehouden, deze klassen hebben een from en to node.
Een vertex classe implementeerd een set met edges.
\textbf{Adjecency Matrix}: Matrix van 2 dimensionale array met integers, row voor uitgaande verbinden en column voor inkomende verbindingen.
1 betekend verbinding, 0 is afwezigheid van verbinding.
Beter geschikt voor een graaf met een dense graphs.

\subsection{Depth-first search}
Vindt een pad tussen een begin en eind node, een voorbeeld is het steeds rechts blijven gaan in een doolhof.
Het algoritme gebruikt een recusieve methode, track bezochte nodes en back-track wanneer er geen mogelijkheden meer zijn.
Bouw het pad terug wanneer de target gevonden is.

\subsection{Breath-first search}
Gaat in plaats van de diepte de breedte in, eerst worden alle neighbours bekeken en daarna pas een level dieper.
In plaats van recursie gebruik je een queue.
