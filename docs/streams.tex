\section{Streams}
Streams zijn een sequentie van elementen dat geaggregeerde operaties ondersteund.
Andere manier om een for each te doen door middel van een compate notatie van pipeline verwerking van een reeks waarden.
De Collectie interface implementeert de \texttt{Stream}, dus het wordt door iedere collectie ondersteund.
\texttt{ParallelStream} geeft een verbetering van performance door multi-threading te gebruiken, maar vereist dat je code thread-safe is.
Er zijn twee soorten methodes voor een stream.
\textbf{Mapper}: dit is een tussenligende steam bewerkingen, en geeft een stream terug.
Zoals, \texttt{filter()} met een functionele interface als parameter.
\textbf{Reducer}: dit is een terinale methode, stopt de stream.
Zoals, \texttt{count()} dat het aantal elementen telt.

\subsection{Lazy}
Er vindt pas een bewerking op de databron plaats wanneer de terminal oparation is begonnen.
Dat betekend dat een element eerst door de complete pipeline gaat voor het volgende element opgevraagd wordt.
Het geheugen efficientie van de meeste stream operations is daarom $O(1)$.
Grote databronnen kunnen verwerkt worden met een lage gehugen voetprint.

\subsection{Method chaining}
Ook bekend als dot-chaining, verbeterd de code beknoptheid en leesbaarheid.
Worden gebruikt door streams voor het flexibel toepassen van combinaties van operaties in elke volgorde.

\subsection{Mapper}
Is een tussenliggende steam bewerking voor het selecteren en transformeren van waardes, levert een andere stream terug.
Methodes: \texttt{map(Function<T, R>)}, \texttt{mapToInt()}, \texttt{mapToDouble()}, \texttt{filter(Predicate<T>)}, \texttt{distinct}, \texttt{limit(maxSize)}, \texttt{flatMap(Function<T, Stream<R>>)}.

\subsection{Reducer}
Is een final method die de waardes verzameld of combineert en terug geeft.
Methodes: \texttt{anyMatch(Predicate<T>)}, \texttt{allMatch(Predicate<T>)}, \texttt{NoneMatch(Predicate<T>)}, \texttt{sum()}, \texttt{count()}, \texttt{reduce(BinaryOperator<T>)}.

Voorbeelden: 
Alle boeken met een author: \texttt{stream().allMatch(b -> b.getAuthors().size() > 0)}.
Totaal nummer van pagina's: \texttt{stream().mapToInt(Book:getNumPages).sum()}.
Maximum number van pagina's: \texttt{steam().mapToInt(Book::getNumPages).max().orElse(0)}

\subsubsection{Reducer methode}
Generic reducer methode zodat je logica zelf kan implementeren, implementeert incrementele gegevensaggregatie.
Gebruikt BinaryOperation, eerste parameter is de uitkomst van het vorige element in de stream en tweede parameter nieuwe element in de stream.
Vind boek met langste titel: \texttt{stream().reduce((b1, b2) -> (b.getTitle().length() > b2.getTitle().length() ? b1 : b2)).get()}
Verschillende signatures:
\textbf{reduce(BinaryOperator<T> accumulator)}: Alleen een accumulator, reducer start leeg.
\textbf{reduce(T identity, BinaryOperator<T> accumulator)}: Identity geeft een start waarde voor accumulator.
\textbf{reduce(U identity, BiFunction<U ? super T, U> accumulator, BinaryOperator<T> combinor)}:
Combinor gebruikt je als het subtotal en het element in de accumulator operatie twee verschillende types zijn.
Of als we een parallel stream stream gebruikenen moeten we met een combinor resultaten combineren in een enkele stream.

\subsection{Collector}
Verzameld, en eventueel verkleind en / of transformeerd, invoer elementen in een container.
Methodes:
\textbf{Steam to Collection}: \texttt{toList()}, \texttt{toSet()}.
\textbf{Stream to String}: \texttt{joining(delimiter, prefix, postfix)}.
\textbf{Stream to Map}: \texttt{toMap(keyMapper, valueMapper, merge)}, \texttt{groupingBy(classifier)}.
\textbf{Statistics calculators}: \texttt{averagingDouble(mapper)}, \texttt{summarizingDouble(mapper)}, \texttt{maxBy(comparator)}, \texttt{minBy(comparator)}.
\textbf{Generic Reducer}: \texttt{reducing(identity, binaryOperator)}.

Voorbeelden:
Maak een set van alle boeken voor 2000:
\texttt{.stream().filter(b -> yearOfIssue < 2000).collect(collectors.toSet())}
Bereken gemiddelde nummer van autheurs per boek:
\texttt{.stream()collect(Collectors.averagingDouble(b -> b.getAuthors().size()))}
Maak een lijst van alle titels van boeken met meer dan drie auteurs:
\texttt{.stream().map(Book::getTitle).collect(Collectors.joining("\\n", "- book titles", "-"))}

\subsubsection{Collect to Map}
Maakt een Map<K,V> van een Stream<T>.

\textbf{toMap()}: 
Specificeer de key en waarde in een funcitie en geeft een functie om ze te mergen.
\texttt{toMap(Function<T,K> keyMapper, Function<T,V> valueMapper, binaryOperator<V> merger)}.
Voorbeeld, maak een lijst van auteurs met het aantal gepubliseerde boeken:
\texttt{.stream().flatMap(b -> b.getAuthors().stream()).collect(Collectors.toMap(a -> a,a -> 1, math::addExact))}.

\textbf{groupingBy()}: vergelijkbaar met \textit{group by} in SQL.
\texttt{groupingBy(Function<T,K> keyMapper, Collector<T,?,V> reducer)}.
Voorbeeld, maak een lijst van alle auteurs met totaal aantal van gepublisheerde boeken:
\texttt{.stream().flatMap(b -> b.getAuthors().stream()).collect(Collectors.groupingBy(a -> a, Collectors.counting()))}

\subsection{Optional<T>}
Is een datatype met \'e\'en optionele waarde, terug geven door sommige reducers.
Als een collectie leeg is of er geen match is in de filter is Optional ook leeg.
Methodes:
\texttt{get()}: geeft waarde als het bestaat, anders exceptie.
\texttt{orElse()}: geeft waard als het bestaat, anders opgegeven waarde.
\texttt{isEmpty()}: false als waarde bestaat, anders true.

\subsection{Primitieve types}
Het verschil tussen Stream<Double> en DoubleStream, het concept is hetzelfde maar implementatie is anders.
\textbf{Stream<Double>}:
Een stream van Double objecten, heeft extra method: \texttt{.collector(Collector)}.
Verander in een DoubleStream: \texttt{.mapToDouble(x -> x)}.

\textbf{DoubleStream}:
Een stream van primitieve van primitieve Doubles, heeft extra method: \texttt{.average()}.
Verander in een Stream<Double>: \texttt{.boxed()}.

\subsection{Sorted streams}
Sorteer een steam met \texttt{.sorted(Comparator<T>)}.
Extra functies: \texttt{dropWhile(Predicate<T>)}, \texttt{.takeWhile(Predicate<T>)}.
Nadeel is dat het streams minder efficient maken omdat elementen eerst gesorteerd moeten worden.

\subsection{Stream operations}
Operaties op de stream moeten niet interferend zijn, niet de born van de stream aanpassen.
In de meeste gevallen staatloos zijn, niet afhankelijk van veranderingen in de staat tijdens uitvoren van de pipeline.
Vermijd bijwerkingen, veranderen van globale data buten de scope van de pipeline.

\subsection{Forks}
Streams kunnen na het uitvoeren van een terminal operations niet opnieuw gebruikt worden.
Bijvoorbeeld geen \texttt{.count()} na dat je \texttt{.sum()} hebt uitgevoerd.
